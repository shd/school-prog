\section{Кортежи (упорядоченные $n$-ки)}

Бывают ситуации, когда нужно запомнить (передать) два числа одновременно.

Давайте напишем функцию, которая возвращает все целые числа, отличающиеся от 
данного целого числа на $1$. 
Скажем, для $4$ функция должна вернуть два числа: $3$ и $5$.
Аналогично, для $-18$ --- числа $-17$ и $-19$.

Содержательно задача тривиальная (по числу $x$ вернуть $x+1$ и $x-1$),
но пока что мы писали функции, возвращающие не более одного числа.
Мы можем написать две функции: одна будет прибавлять 1, а другая --- вычитать,
но есть другой способ.
, называемый кортежем или
упорядоченной $n$-кой.

Сперва научимся создавать значения этого типа. Для этого нужно перечислить
значения, которые вы хотите положить в кортеж, через запятую и окружить 
скобками:

Примеры:
\begin{verbatim}
let a = (1,2);;
let b = (1,3,"asdf",9.);;
let c = ((1,2),(3,4));;
\end{verbatim}
Заметим, что константа c хранит кортеж из двух других кортежей. Это, 
естественно, вполне допустимо.

Кортежи также можно указывать в сопоставлении с образцом:

\begin{verbatim}
type roots = One of float | Any | None

let roots_info t =
  match t with
    (Any, _) -> "бесконечное количество корней"
  | (_, Any) -> "бесконечное количество корней"
  | _        -> "конечное количество корней";;

print_string (roots_info (One 1., Any));;
\end{verbatim}

И, конечно, кортежи можно указывать как параметры в алгебраических типах:

\begin{verbatim}
type roots2 = Two of float*float | One of float | Any | None
\end{verbatim}

\subsection{Формальная грамматика}

\begin{bnf}\begin{eqnarray*}
\n{<сигнатура-кортежа>} &::=& \n{<сигнатура-типа>} 
  \left\{ \s{*} \n{<сигнатура-типа>} \right\}^+ \\
\n{<литерал-кортежа>}   &::=& 
  \s{(} \n{<выражение>} \left\{\s{,} \n{<выражение>} \right\}^+ \s{)}\\
\n{<образец-кортежа>}   &::=& \s{(} \n{<образец>} \left\{ \s{,} \n{<образец>}\right\}^+ \s{)}
\end{eqnarray*}\end{bnf}

\section{Рекурсия}

Что можно сказать про такой тип?
type a = S of a | Nil;;

Отличие его от обычного алгебраического типа - использование своего имени 
внутри описания. Выглядит, возможно, довольно непривычно для вас, но если
будем действовать по порядку, все получится. Воспользуемся нашим упрощенным
определением (тип - это множество значений), и попробуем найти значения,
которые этому типу соответствуют.

\begin{itemize}
\item Давайте сперва посмотрим на конструктор \verb!Nil!. 
У \verb!Nil! нет параметров, поэтому значение \verb!Nil!, очевидно, принадлежит 
типу \verb!a!.
\item Теперь будем разбираться с конструктором \verb!S!. Этот конструкторв имеет 
параметр типа \verb!a!, значит, если \verb!x! - это значение типа \verb!a!, 
то \verb!S(x)! - это тоже значение типа \verb!a!.

\item Но раз так, то и \verb!S(S(Nil)! -- типа \verb!a!.
И вообще, любое выражение вида \verb!S(S(...S(Nil)...))! -- типа \verb!a!. 
Но, поскольку кроме рассмотренных конструкторов \verb!S! и \verb!Nil! 
никаких других способов 
образовать тип a в его описании не указано, у него нет и других значений. 
Таким образом, мы исчерпывающе описали тип a.
\end{itemize}

Теперь разберемся, как нам, например, напечатать значение этого типа.
Первая идея, приходящая в голову, не сработает:

\begin{verbatim}
let print_a v = 
  match v with
    Nil       -> print_string "Nil"
  | S(Nil)    -> print_string "S(Nil)"
  | S(S(Nil)) -> print_string "S(S(Nil))"
  ...
\end{verbatim}
поскольку значений типа бесконечно много.

Поэтому нам потребуется новое понятие - рекурсивная функция. Как видно из
названия, это функция, которая вызывает сама себя. Вот как будет выглядеть
рекурсивная функция печати этого типа:

\begin{verbatim}
let rec string_of_a v =
  match v with
    Nil  -> "Nil"
  | S(x) -> "S(" ^ string_of_a x ^ ")";;

let print_a v = print_string (string_of_a v);;
\end{verbatim}

Обратим внимание на две тонкости в описании \s{string\_of\_a}. 

Во-первых, после ключевого слова \s{let} появилось ранее не встечавшееся ключевое 
слово \s{rec}. Это слово необходимо писать при описании всякой рекурсивной функции. 
Во-вторых, в четвертой строке функции \s{string\_of\_a} происходит ее 
рекурсивный вызов --- вызов описываемой функции.

Возможно, вам непонятно, что значит, что функция вызывает себя саму. 
Действительно, внешне это немного напоминает барона Мюнгхаузена, 
вытаскивающего себя за волосы из болота. 
Поступим в точности как со значениями --- будем разбираться по отдельности.

Для удобства повторим код еще раз:

\begin{verbatim}
let rec string_of_a v =
  match v with
    Nil  -> "Nil"
  | S(x) -> "S(" ^ string_of_a x ^ ")";;
\end{verbatim}

\begin{enumerate}

\item
Чему равно \s{string\_of\_a\ Nil}? Ответ: \s{"Nil"}, поскольку эта
ветвь вычислений не использует рекурсии. Давайте запомним это и более не 
будем думать об этом случае. Для четкости отразим это в таблице:

\begin{tabular}{ll}\\
\hline
Выражение&Результат вычисления\\
\hline
\s{string\_of\_a\ Nil}&\s{"Nil"}\\
\hline\\
\end{tabular}

\item

Чему равно \s{string\_of\_a (S(Nil))}? Проследив выполнение конструкции
\s{match}, видим, что это будет 
\s{"S(" \^\ string\_of\_a x \^\ ")"}. 

Мы столкнулись с рекурсивным вызовом,
но из таблицы мы ведь знаем результат конкретно этого рекурсивного вызова,
ведь \s{(string\_of\_a Nil)} - это \s{"Nil"}.
Поэтому мы можем этот результат подставить:
\s{"S(" \^\ "Nil" \^\ ")"}, то есть \s{"S(Nil)"}.
Отразим и этот факт
в таблице:

\begin{tabular}{ll}\\
\hline
Выражение&Результат вычисления\\
\hline
\s{string\_of\_a\ Nil}&\s{"Nil"}\\
\s{string\_of\_a\ (S(Nil))}&\s{"S(Nil)"}\\
\hline\\
\end{tabular}

\item
Чему равно \s{string\_of\_a\ (S(S(Nil)))}? 
Повторим рассуждение: это 
\s{"S(" \^\ string\_of\_a\ (S(Nil)) \^\ ")"}, то есть
\s{"S(S(Nil))"}, что мы снова можем запомнить:

\begin{tabular}{ll}\\
\hline
Выражение&Результат вычисления\\
\hline
\s{string\_of\_a\ Nil}&\s{"Nil"}\\
\s{string\_of\_a\ (S(Nil))}&\s{"S(Nil)"}\\
\s{string\_of\_a\ (S(S(Nil)))}&\s{"S(S(Nil))"}\\
\hline
\end{tabular}
\end{enumerate}

Легко видеть, что так мы разберем все возможные случаи. Ведь каждое 
значение типа \s{a} имеет конечное число вложенных значений \s{S}, после 
которых, в самой глубине, хранится значение \s{Nil}. Наше же рассуждение, 
идя в обратную сторону, неизбежно дойдет до значения с любым конечным 
количеством букв \s{S}.

Конечно, эти рассуждения не полностью отражают реальный процесс вычислений, 
компьютер не запоминает результат функций, но для первого знакомства с 
рекурсивными функциями такого понимания должно быть вполне достаточно.

\subsection{Натуральные числа}

Напомним, как в аксиоматике Пеано представляются натуральные числа:
постулируется существование константы $0$ и операции прибавления $1$ 
(\emph{инкремента}). 
В формулах мы будем обозначать инкремент штрихами:
число $a''$ --- это число $a$, увеличенное на 1 два раза (то есть $a + 2$), 
а число $3$ представимо как $0'''$. 
Что интересно, этих простых операций достаточно, чтобы определить все 
операции целочисленной арифметики (сложение, вычитание, 
умножение).

\begin{tabular}{ll}\\
\hline
Операция&Определение\\
\hline
Инктремент&$\n{inc} (a) = a'$\\
Сложение&$\n{plus} (0,b) = b$; $\n{plus} (a',b) = (\n{plus} (a,b))'$\\
Декремент&$\n{dec} (a') = a$; $\n{dec} (0) = 0$\\
Вычитание&$\n{minus} (a',b') = \n{minus} (a,b)$; $\n{minus} (0,a) = 0$; $\n{minus} (a,0) = a$\\
Умножение&$\n{mul} (a',b) = \n{plus} (b, \n{mul} (a,b))$; $\n{mul} (0,b) = 0$\\
\hline
\end{tabular}

\begin{example}
Давайте вычислим $3 \cdot 2$ с помощью этих определений. 
Заметим, что 3 --- это $0'''$, а 2 --- это $0''$. Тогда вычисление
$\n{mul} (0''', 0'')$ будет состоять из следующих преобразований:

\begin{tabular}{ll}\\
\hline
Выражение в арифметике Пеано&Обычная запись\\
\hline
  $\n{mul} (0''', 0'')$ = & $3 \cdot 2$\\
  $\n{plus} (0'', \n{mul} (0'', 0''))$ = & $2 + (2\cdot 2)$\\
  $\n{plus} (0'', \n{plus} (0'', \n{mul} (0', 0'')))$ = & $2 + (2 + (1\cdot 2))$\\
  $\n{plus} (0'', \n{plus} (0'', \n{plus} (0'', \n{mul} (0, 0''))))$ = & $2 + (2 + (2 + (0\cdot 2)))$\\
  $\n{plus} (0'', \n{plus} (0'', \n{plus} (0'', 0)))$ = & $2 + (2 + (2 + 0))$\\
  $\n{plus} (0'', \n{plus} (0'', (\n{plus} (0', 0))'))$ = & $2 + (2 + ((1 + 0) + 1)$\\
  $\n{plus} (0'', \n{plus} (0'', (\n{plus} (0, 0))''))$ = & $2 + (2 + ((0 + 0)) + 2)$\\
  $\n{plus} (0'', \n{plus} (0'', 0''))$ = & $2 + (2 + 2)$\\
  $\n{plus} (0'', (\n{plus} (0', 0''))')$ = & $2 + ((1 + 2) + 1)$\\
  $\n{plus} (0'', (\n{plus} (0, 0''))'')$ = & $2 + ((0 + 2) + 2)$\\
  $\n{plus} (0'', 0'''')$ = & $2 + 4$\\
  $(\n{plus} (0', 0''''))'$ = & $(1 + 4) + 1$\\
  $(\n{plus} (0, 0''''))''$ = & $(0 + 4) + 2$\\
  $0''''''$ & $6$\\
\hline\\
\end{tabular}
\end{example}

Математическую сторону этих определений (доказательство корректности и т.п.)
мы опустим, а вот на программистскую как раз и обратим внимание.
Ведь тип данных \s{a}, описанный выше, и является как раз типом данных,
представляющим натуральные числа <<в стиле аксиоматики Пеано>>.

Покажем, например, как реализовать операцию инкремента, это очень просто:
\begin{verbatim}
let inc x = S x;; (* дописываем одну букву S - прибавляем один *)
\end{verbatim}

Чуть посложнее операция сложения, там в определении есть два случая.
Первый случай --- когда первое слагаемое равно 0, и второй случай, когда
оно содержит применение как минимум одной операции прибавления 1 ($'$). 
Впрочем, и здесь мы можем впрямую следовать определению.
\begin{verbatim}
let rec plus a b = 
  match a with
    Nil -> b
  | S a -> S (plus a b);;
\end{verbatim}

А так реализуется вычитание:
\begin{verbatim}
let rec minus a1 b1 = 
  match (a1, b1) with  (* разбор случаев из определения *)
    (S a, S b) -> minus a b
  | (Nil, a) -> Nil
  | (a, Nil) -> a;;
\end{verbatim}

И завершит этот пример функция преобразования значений типа \s{a} в 
значения типа \s{int}:

\begin{verbatim}
let rec int_of_a v =
  match v with
    Nil -> 0
  | S(x) -> 1 + int_of_a x;;
\end{verbatim}

Здесь мы действуем впрямую по определению: если значение типа \s{a}
имеет вид \s{S(x)} --- прибавление 1 к $x$ --- то надо вычислить значение
для \s{x}, а затем прибавить к нему 1.

\subsection{Списки}

Мы уже познакомились с одним частным случаем списка --- типом \s{a} из 
предыдущего
параграфа. Теперь настала пора познакомиться с ними в общем случае.

Вообще, список можно было бы описать примерно так:
\begin{verbatim}
type 'a list = Cons of 'a * 'a list | Nil
\end{verbatim}

Как видите, это тип похож на тип \s{a}, главное его отличие --- этом
непонятном символе \s{'a} перед именем \s{list}, а также в наличии 
дополнительного значения в каждом элементе \s{Cons}.
Естественно, значения тоже похожи:
\begin{verbatim}
Cons (1, Cons (2, Cons (3, Nil)))
\end{verbatim}

В силу частоты использования, тип список встроен в язык,
и конструкторы типа \s{Cons} и \s{Nil} имеют специальные
имена: \s{::} и \s{[]} соответственно, а вместо 
\s{Cons (1, Cons (2, Cons (3, Nil)))}
надо писать \s{1 :: 2 :: 3 :: []}
Также, есть еще один вариант записи этого списка:
\s{[1;2;3]}, 

Естественно, эти сокращения так же применяются и в сопоставлении с образцом.

Важно! Конструктор типа \s{::} несимметричен. Слева от него всегда должен
быть указан элемент, а справа --- список, например так: \s{1 :: []}. 
Выражение \s{[] :: 1} некорректно.

Но при этом \s{[1]::[[1]]}, так же, как и \s{[1]::[]} --- корректные списки,
состоящие из списков целых чисел.

Пример (подсчет длины списка, то есть количества \verb!::!):

\begin{verbatim}
let rec list_length l =
  match l with
    [] -> 0
  | l1::ls -> 1 + list_length ls;;
\end{verbatim}

В том случае, если список состоит только из \s{[]}, он имеет длину 0.
Если список - это \s{l1::[]}, то длина равна $1 + \s{list\_length\ []}$, то есть 1.
Если список - это \s{l2::l1::[]}, то длина - $1 + \s{list\_length\ l1::[]}$, то есть 2.
И в общем случае, если есть список \s{l1::ls}, то его длина равна
$1 + \s{list\_length\ ls}$.

Список может хранить элементы любого типа, но этот тип должен быть
для всех элементов один и тот же.
Тип списка \s{[1;2;3]} --- \s{int list}, тип списка \s{["1";"a";"c"]} --- 
\s{string list},
список \s{[[1];[2];[]]} имеет тип \s{int list list} (для ясности можно поставить
скобки: \s{(int list) list}), но списка \s{[1;"2"]} быть не может.

%<сигнатура-типа-списка> ::= <сигнатура-типа> list
%<конструктор-cons-для-списка> ::= <выражение> :: <выражение>
%<конструктор-nil-для-списка> ::= []
%<литерал-списка> ::= [ [<выражение> (; <выражение>)*] ]



\subsection{Сигнатура типа}

\emph{Сигнатурой} типа мы будем называть его обозначение на языке Окамль.
Сигнатуры элементарного типа нам уже знакомы:

\begin{tabular}{ll}
\hline
Тип&Сигнатура типа\\
\hline
строчки            &string\\
целые числа        &int\\
плавающие числа    &float\\
булевские (логические) значения &bool\\
одноэлементный тип &unit\\
\hline
\end{tabular}

Но мы уже умеем строить значения не только этих простых типов, но и более 
сложных конструкций: речь идет о функциях.

Рассмотрим следующую функцию:
\begin{verbatim}
let x a b = if a > 0 then b else 1.;;
\end{verbatim}

Эта функция берет два аргумента (целочисленный и плавающий) и возвращает 
плавающее число. Соответственно, сигнатура этой функции имеет следующий вид:
\begin{verbatim}
x: int -> float -> float
\end{verbatim}
Первая из стрелок \s{->}, указанная в сигнатуре, разделяет сигнатуры типов
аргументов функции, а вторая отделяет тип аргумента от типа результата функции.

Как можно догадаться, какой тип у функции?
Когда мы пишем функцию, мы уже должны иметь представление, аргументы каких 
типов ей передаются и какой тип значений она должна возвращать.
Ведь мы же знаем, что она должна делать.
Тем не менее компилятор не знает о наших предположениях, он судит по 
тому коду, который мы написали --- и, основываясь на нем, автоматически
выводит типы аргументов и результата. Попробуем сделать это и мы.

Это как головоломка. Вот есть такой код:
\begin{verbatim}
let z x y z = if (x + y) / 2 > 0 then "false" else z;;
\end{verbatim}
Что можно по нему заключить?
\begin{enumerate}
\item Формальные параметры \s{x} и \s{y} складываются между собой
операцией \s{+}, которая требует, чтобы левый и правый операнды
имели тип \s{int}. Значит, \s{x} и \s{y} могут иметь только тип \s{int}
--- иначе код содержит ошибку.
\item Тип результата --- \s{string}, поскольку результат вычисления 
операции \s{if} является результатом всей функции --- а одна из ветвей
возвращает значение \s{"false"}, имеющего тип \s{string}.
\item И наконец ветви операции \s{if} должны иметь одинаковый тип ---
значит, тип результата совпадает с типом параметра \s{z}.
\item Итого, аргументы \s{x} и \s{y} типа \s{int}, аргумент \s{z} типа
\s{string}, и результат функции --- тоже типа \s{string}.
Значит, сигнатура типа функции \s{z} такова: 
\s{int\ ->\ int\ ->\ string\ ->\ string}.
\end{enumerate}

Рассмотрим еще несколько примеров:

\begin{tabular}{ll}\\
\hline
Значение&Сигнатура типа\\
\hline
\s{let v () = "это строка";;}                      &\s{v: unit -> string}\\
\s{let w () = print\_string "Привет\n";;}&\s{w: unit -> unit}\\
\s{let x a b = (a = 0) || (b = 0.);;}              &\s{x: int -> float -> bool}\\
\s{let y a b c = (a = "{}"{}) || (b = 0) || c;;}       &\s{y: string -> int -> bool -> bool}\\
\s{let z x = if x = "{}4"{} then 1 else -1;;}          &\s{z: string -> int}\\
\hline\\
\end{tabular}

