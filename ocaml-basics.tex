\documentclass[12pt,a4paper,oneside]{book}
\usepackage[utf8]{inputenc}
\usepackage[english,russian]{babel}
%\usepackage{amsmath}
\usepackage{amsthm}
\usepackage{amssymb}
\usepackage{enumerate}
\usepackage{bnf}
%\usepackage[UglyObsolete,tight,heads=LaTeX] {diagrams}
\usepackage[left=2cm,right=2cm,top=2cm,bottom=2cm,bindingoffset=0cm]{geometry}
\usepackage{proof}
\begin{document}

\theoremstyle{definition}
\newtheorem{definition}{Определение}[section]
\newtheorem*{example}{Пример}
\theoremstyle{theorem}
\newtheorem{theorem}{Теорема}[section]
\newtheorem{lemma}[theorem]{Лемма}

\newcommand{\s}[1]{\texttt{#1}}
\newcommand{\xl}{$\lambda$}
\newcommand{\+}{\lambda}
\newcommand{\bredmath}{\ \longrightarrow_\beta\ }
\newcommand{\bred}{$\bredmath$}
\newcommand{\mbred}{$\ \longrightarrow\!\!\!\!\rightarrow_\beta\ $}
\newcommand{\lid}[1]{\textit{#1}}
\newcommand{\concat}{\hat{\ \ }}

\def\from#1{\par \parbox{0.7\textwidth}{\par \hfill\raggedleft \it #1}} 

\newenvironment{epigraph}% 
{\begin{list}{}{\setlength{\leftmargin}{0.3\textwidth}}\item[]}% 
{\end{list}} 

\chapter*{Введение в язык Окамль}

\setcounter{chapter}{1}

\newcommand\n[1]{\textit{#1}}

\section {Структура программы}
Поскольку мы только учимся, давайте несколько упрощённо скажем, что программа 
на языке Окамль --- это последовательность выражений и описаний,
указываемых через двойное двоеточие (;;). Или то же самое зададим грамматикой:

\begin{bnf}\begin{eqnarray*}
\n{<программа>} &::=& ( \n{<описание>} | \n{<выражение>} ) \s{;;} [ \n{<программа>} ]
\end{eqnarray*}\end{bnf}

\subsection{Выражение}

Выражения как идея вам должны быть хорошо знакомы.
Например, вы наверняка умеете строить арифметические выражения с помощью арифметических действий.
Выражение $2 + 2$ имеет в Окамле такой же смысл, как и в арифметике (вычисление
суммы двух чисел).

Однако, действия (и, соответственно, выражения, которые из них составляются) бывают не только 
арифметические. Рассмотрим такую программу:

\begin{verbatim}
print_string "Здравствуй, мир!";;
\end{verbatim}

Здесь выражение содержит действие \s{print\_string} --- печати строки на экране.
При запуске эта программа печатает текст \s{Здравствуй, мир!}, после чего заканчивает работу. 

Давайте напишем что-нибудь еще:

\begin{verbatim}
print_string "Здравствуй, ";;
print_string "мир!";;
\end{verbatim}

Эта программа делает в точности то же самое, но
устроена сложнее: в ней два выражения, каждое из которых
печатает свой кусок текста.

В целом, возможные действия в Окамле очень разнообразны.
% --- арифметических, вызовов функций и т.п.

\subsection{Описание}

Идея описаний для вас тоже не должна быть новой. Описания подобны словам
<<обозначим за $v$ скорость пешехода>>, часто встречающимся в решениях задач
по физике и математике, они позволяют давать выражениям имена.
Зачем делают описания? Чтобы разбить выражения на части, упростить их 
запись, сделать их понятными.

Если вы хотите дать какому-нибудь выражению имя (иначе называемое как 
\emph{идентификатор}), делайте это в соответствии с грамматикой:

\begin{bnf}\begin{eqnarray*}
\n{<описание>} &::=& \s{let } \n{<идентификатор>} \s{ = } \n{<выражение>}
\end{eqnarray*}\end{bnf}

Рассмотрим, например, такую программу:

\begin{verbatim}
let str = "Здравствуй, мир!";;
print_string str;;
\end{verbatim}

В данной программе в первой строке создается константа \s{str}, 
содержащая строчку \s{"Здравствуй мир!"} (иными словами, это описание 
константы \s{str}), которая затем печатается в выражении во второй строке. 

Раз это константа, то ее можно использовать сколько угодно раз:
\begin{verbatim}
let str = "Здравствуй, мир!";;
print_string str;;
print_string str;;
\end{verbatim}
Данная программа будет печатать строчку два раза.

Константу можно переопределить:
\begin{verbatim}
let str = "Здравствуй, мир!";;
print_string str;;
let str = "Привет, мир!";;
print_string str;;
\end{verbatim}
При этом старое значение константы теряется. Данная программа 
напечатает следующий текст:
\begin{verbatim}
Здравствуй, мир!Привет, мир!
\end{verbatim}

Идентификаторы констант строятся по следующим приавилам:
\begin{bnf}\begin{eqnarray*}
\n{<идентификатор>}    &::=& \n{<начальный символ>} \{\n{<символ>}\}*\\
\n{<начальный символ>} &::=& \s{a}..\s{z} | \s{\_}\\
\n{<символ>}           &::=& \s{a}..\s{z} | \s{A}..\s{Z} | \s{0}..\s{9} | \s{\_} | \s{'}
\end{eqnarray*}\end{bnf}

Таким образом, идентификаторы \s{fAAA9\_'} и \s{\_\_\_} вполне допустимы, тогда как
\s{0ffa}, \s{'abcd}, \s{ABCD} не являются допустимыми идентификаторами констант.

%Надо отметить, что имя функции \s{print\_string} само является идентификатором
%константы, только константы не строкового типа, а функционального.

\subsection{Пробелы, переводы строки, комментарии}

Для компилятора Окамля существенно, чтобы два идентификатора 
отделялись всегда пробелами. Также, не смотря на то, что синтаксис языка не 
требует указания пробела между знаками операций и идентификаторами
(скажем, мы можем написать \s{let x="мир";;}, мы можем их разделить
их пробелом, если это нам кажется правильным: \s{let x = "мир" ;;}
Везде же, где может/должен находиться хотя бы один 
пробел, вместе с ним (или вместо него) может также находиться произвольное 
количество пробелов и/или переводов строк, а также \emph{комментариев}.

Комментарий --- это произвольный текст, ограниченный символами 
\s{(*} и \s{*)}, который не анализируется компилятором. В комментариях
обычно пишут какие-то соображения на естественном языке, 
комментирующие код программы. 

Код программы с комментариями может выглядеть так:

\begin{verbatim}
print_string    "Здравствуй, ";; (* Здравствуй, мир! - это перевод с 
                                    английского классического текста *)
print\_string                    (* hello, world *)
   "мир!"
;;		 
\end{verbatim}

Ещё комментарии используются для временного исключения части кода. Перед
тем, как удалить код из программы, бывает, стоит его <<закомментировать>>. 
Вдруг передумаем удалять?

\section{Элементарные типы данных: числа и строки}

Из школьной программы у вас должно уже сложиться интуитивное понятие типа
значения: скажем, скорость измеряется в километрах в час, а масса ---
в килограммах. При этом ясно, что выражение <<1 км/ч>> + <<5 кг>> 
в общем случае бессмысленно и никакого естественного способа его вычислить 
мы не знаем.

В программировании это интуитивное понятие встречается очень часто, поэтому,
чтобы двигаться дальше, нам потребуется его формализовать. Данная 
формализация несовершенна, но подходит для целей знакомства с
языком Окамль.

\begin{definition}
\emph{Типом данных} называется множество значений. 
\end{definition}

Например, целочисленный тип данных в языке Окамль 
(при работе на 32-разрядном компьютере) составляют все целые 
числа в диапазоне от $-2^{30}$ до $2^{30}-1$.

Тип может иметь имя, для целых чисел в Окамле используется имя \s{int},
для чисел с плавающей точкой --- \s{float}, для строчек --- \s{string}.

\emph{Литералом} некоторого значения мы будем называть его запись
в программе. 
Например, литералом для строчки \s{Здравствуй, мир!} 
будет \s{"Здравствуй, мир!"}.

\subsection{Строки}

Рассмотрим строковые литералы подробнее:
\begin{bnf}\begin{eqnarray*}
\n{<строковый-литерал>} &::=& \s{"{}} \n{<символ>}* \s{"{}} \\
\n{<символ>} &::=& \n{<печатный символ>} | \s{\textbackslash{}} \n{<спецсимвол>}\\
\n{<спецсимвол>} &::=& \s{"{}} | \s{\textbackslash{}} | \s{n}
\end{eqnarray*}\end{bnf}Здесь \n{<печатный символ>} 
--- это любой символ, кроме двойных кавычек (") и обратной косой черты (\textbackslash{}). 

Зачем нужны спецсимволы? Например, чтобы задавать двойные кавычки как часть текста.
Рассмотрим текст \verb!""^""!: следует ли его понимать как соединение двух пустых
строк \verb!("")^("")! или его следует понимать как литерал строки \verb!"^"!?
Иными словами, вторые и третьи кавычки --- это символы строки или ограничители строки?

Чтобы избежать путаницы, в Окамле принято считать, что двойные кавычки всегда ограничивают
строку, а для указания кавычек как символа используется последовательность
из двух символов: обратной косой черты и кавычек (\verb!\"!). Литерал 
\s{"\textbackslash{}"\textbackslash{}\textbackslash{}"} задает строчку 
из кавычки, за которой идет обратная косая черта, а литерал 
\s{"\textbackslash{}\textbackslash{}\textbackslash{}"\textbackslash
{}\textbackslash{}\textbackslash{}""} задает 
строчку \s{\textbackslash{}"\textbackslash{}"}.

Но помимо обычных символов, существуют и другие символы, которые нужно 
иметь возможность
вставлять в строки, но которым не соответствует никакого изображения. 
Нам пока потребуется только один из них --- символ перевода (окончания) 
строки. Всё, что печатается после него, будет печататься
на новой строке. Для его задания служит последовательность 
\s{\textbackslash{}n}. 

Так, программа 
\begin{verbatim}
print_string "Здравствуй,\nмир!";;
\end{verbatim}

напечатает 2 строки:
\begin{verbatim}
Здравствуй,
мир!
\end{verbatim}

\subsection{Целые числа}

Литералы целых чисел задаются следующей грамматикой:

\begin{bnf}\begin{eqnarray*}
\n{<литерал целого числа>} &::=& \{\s{0}..\s{9}\}^+
\end{eqnarray*}\end{bnf}

\begin{example} 
Следующие строчки являются литералами целых чисел:
\begin{center}
\s{2419}\quad\s{71000}\quad\s{001}\quad\s{0}
\end{center}
\end{example}

%Сами целые числа, как уже указывалось, в случае 32-разрядной ОС имеют
%диапазон от $-2^30$ до $2^30-1$, но чтобы задать отрицательное число,
%требуется применить унарную операцию \emph{минус}: так, число \emph{-23}
%может быть записано как $-23$. Хоть эта запись и выглядит неотличимо
%от традиционной записи отрицательного числа, есть ситуации, где эта
%разница будет существенной.

\subsection{Плавающие числа}

\emph{Плавающие числа} (или, более формально, \emph{числа с плавающей запятой})
предназначены для представления дробных и вещественных чисел.
К сожалению, компьютер не может представить многие вещественные числа точно,
только приближённо, отсюда и иное название.

Упрощённый синтаксис литералов чисел с плавающей запятой:
\begin{bnf}\begin{eqnarray*}
\n{<литерал плавающего числа>} &::=& {\{\s{0}..{9}\}^+}.\{\s{0}..\s{9}\}^+ \\
                     &|& {\{\s{0}..\s{9}\}^+}. \\
                     &|& .\{\s{0}..\s{9}\}^+
\end{eqnarray*}\end{bnf}
\begin{example} 
Примеры литералов плавающих чисел:
\begin{center}
\s{00024.19}\quad\s{71.000}\quad\s{0.17}\quad\s{147.}\quad\s{.1717}
\end{center}
\end{example}

Как видите, от обычной записи десятичной дроби отличий два:
\begin{itemize}
\item Вместо традиционной запятой в числе используется точка.
\item Помимо основной формы есть сокращенные, когда ноль в целой или дробной
части может быть опущен: числа \s{0.0}, \s{0.} и \s{.0} --- это одно и то 
же число.
\end{itemize}

%Чтобы записать отрицательное число, надо использовать унарную операцию
%\emph{плавающий минус}: \s{-.14.1} --- это $-14.1$.

\subsection{Арифметические выражения}

Как вы, возможно, обратили внимание, литералы для значений каждого типа 
существенно различаются. Если значение в кавычках --- это строка, если
оно состоит только из цифр --- это целое число, если оно содержит точку ---
это плавающее число.
Такое разделение создано намеренно: никакое значение одного элементарного 
типа не может принадлежать какому-то другому элементарному типу.

Арифметические функции для работы с целыми и плавающими
числами также имеют разные имена. Нельзя пользоваться целочисленной
операцией для работы с плавающими числами и наоборот, плавающей операцией
для работы с целыми.
В таблице приводятся обозначения для четырех основных арифметических 
операций и примеры работы с ними.

\begin{tabular}{lccl}\\
\hline
Действие & Целые & Плавающие & Примеры\\
\hline
Сложение & \s{+} & \s{+.} & \s{14+12}\quad\s{14.+.12.}\\
Вычитание & \s{-} & \s{-.} & \s{41-20}\quad\s{.23-.71.}\\
Умножение & \s{*} & \s{*.} & \s{24*90}\quad\s{78.42*.84.0}\\
Деление & \s{/} & \s{/.} & \s{64/16}\quad\s{0.14/.9881.}\\
\hline\\
\end{tabular}

\begin{example}
Такие выражения содержат ошибки:
\begin{center}
\s{24+42.071}\quad\s{5.0*.5}\quad\s{20/.20}
\end{center}
А такие выражения допустимы:
\begin{center}
\s{24.+.42.071}\quad\s{5*5}\quad\s{20/20}
\end{center}
\end{example}

Точка у арифметических операций для плавающих
чисел --- это часть их имени, как и точка в обозначении соответствующих
литералов. Если вы описываете константу, точки в ее имени \emph{не} ставятся.
Например, в коде ниже три константы для плавающих чисел: \verb!pi!, \verb!diameter!, \verb!area!,
ни при одной из них не указывается точка:
\begin{verbatim}
let pi = 3.1415926;;
let diameter = 4.0;; (* 4 метра *)
let area = pi *. diameter *. diameter /. 4.;; (* вычисляем площадь *)
\end{verbatim}

\section{Условное выражение}

Вернемся к одной из задач: будем решать линейное уравнение вида $a \cdot x + b = c$.
Нетрудно видеть, что это уравнение почти всегда имеет ровно одно решение,
кроме случая, когда $a = 0$. В этом случае, в зависимости от констант $b$ и 
$c$, уравнение либо имеет бесконечно много решений, либо ни одного.

Надо как-то выделить этот случай, поскольку применение классической формулы
$x = (c - b)/a$ даст деление на $0$, что приведет к ошибке выполнения 
программы. В этом нам поможет условное выражение.

\subsection{Булевский тип}

Для начала мы введем новый тип данных --- \emph{булевский (логический, истинностный)} --- 
имеющий два значения \s{false} и \s{true} (\emph{ложь} и \emph{истина} 
соответственно). Данный тип в Окамле имеет идентификатор \s{bool}.

Тип назван так в честь Джорджа Буля --- английского математика, 
предложившего использовать алгебраические методы в логических рассуждениях.
Это ему мы в значительной степени обязаны идеей представлять ложь как $0$,
а истину как $1$.

Для работы с булевскими типами определены несколько операций:

\begin{tabular}{lcl}\\
\hline
Операция & Обозначение & Пример\\
\hline
Логическое <<И>> & \s{\&\&} & \s{true \&\& false}\\
Логическое <<Или>> & \s{{|}|} & \s{false {|}| false}\\
Отрицание & \s{not} & \s{not true}\\
\hline\\
\end{tabular}

\subsection{Условное выражение}

Условное выражение --- это выражение, задаваемое следующей грамматикой:
\begin{bnf}\begin{eqnarray*}
\n{<условное выражение>} &::=& \s{if}\ \n{<условие>}\ \s{then}\ \n{<ветка then>}\ [ \s{else}\ \n{<ветка else>} ]\\
\n{<условие>} &::=& \n{<выражение>}\\
\n{<ветка then>} &::=& \n{<выражение>}\\
\n{<ветка else>} &::=& \n{<выражение>}
\end{eqnarray*}\end{bnf}В том случае, когда ветка else не указана,
предполагается что это сокращение для записи 
\begin{center}
$\s{if} \n{<условие>} \s{then} \n{<ветка then>} \s{else ()}$
\end{center}

Результат вычисления условного выражения зависит от условия.
Если результат вычисления условия --- \s{true}, то результат всего 
выражения --- это результат вычисления ветки then, иначе --- результат 
вычисления ветки else.

Например, результатом выражения ниже будет 3:

\begin{verbatim}
if 2 > 1 then 3 else 44
\end{verbatim}

\subsection{Операции сравнения}

Определим сравнения в Окамле подробнее:

\begin{tabular}{lcl}\\
\hline
Операция & Обозначение & Пример\\
\hline
равно            & \s{=}  & \s{1 = 1}\\
не равно         & \s{<>} & \s{"строка"{} <> "еще одна строка"}\\
больше           & \s{>}  & \s{2 > 1}\\
меньше           & \s{<}  & \s{1.0 < 2.0}\\
больше или равно & \s{>=} & \s{1 >= 1}\\
меньше или равно & \s{<=} & \s{false <= true}\\
\hline\\
\end{tabular}

Все приведенные выше примеры выдают значение \s{true}. Напротив, 
выражения
\begin{center}
\s{1=0}\quad\s{2<>2}\quad\s{"я"{}<"а"}\quad\s{2.0>2.0}\quad\s
{1>=2}\quad\s{true<=false}
\end{center}все имеют значение \s{false}.

Обратите внимание, что операции сравнения, в отличие от арифметических,
берут аргументы произвольного типа. Единственным условием является то, что
тип левого аргумента и тип правого аргумента должен быть одинаков.
Такие операции называют \emph{полиморфными}. 

\subsection{Пример}

Итак, напишем программу, решающую линейное уравнение:

\begin{verbatim}
print_string "x:\n";;
if a <> 0. then 
  print_float ((c -. b) /. a)
else 
  print_string "либо ни одного, либо бесконечно много";;
\end{verbatim}

Но ответ <<либо ни одного, либо бесконечно много>> слишком неопределенный,
ведь мы можем ответить на вопрос точнее.
Добавим второе условное выражение, разделяющее случаи $b = c$ и $b \ne c$.

\begin{verbatim}
print_string "x:\n";;
if a <> 0. then 
  print_float ((c -. b) /. a)
else (if b = c then
  print_string "любое"
else
  print_string "решений нет");;
\end{verbatim}

Обратите внимание на скобки вокруг второго условного выражения. 
С условными выражениями можно обращаться так же, как и с арифметическими, 
в том числе заключать в скобки. 
Впрочем, в данном конкретном случае мы можем написать
это же чуть красивее:

\begin{verbatim}
print_string "x:\n";;
if a <> 0. then 
  print_float ((c -. b) /. a)
else if b = c then
  print_string "любое"
else
  print_string "решений нет";;
\end{verbatim}

%В так написанной программе результатом вычисления обоих условных выражений 
%будет побочный эффект и значение типа \s{unit}. Но все может быть и иначе: 
%в следующем примере вложенное условное выражение имеет тип \s{string}
%и не имеет побочных эффектов.
%
%\begin{verbatim}
%if a = 0. then print_string (if b = c then "любое" else "решений нет")
%          else print_float ((c -. b) /. a);;
%\end{verbatim}

\section{Функции}

Мы уже довольно давно знакомы с функциями (напомним, первая программа 
состояла из вызова функции \s{print\_string}), но только со стороны
пользователя. Сейчас мы научимся функции описывать.

Существует математические определения функции, как отображения, ставящего
в соответствие каждому элементу одного множества (называемого множеством
отправления), ровно один элемент другого множества (называемого множеством
прибытия).

Это определение сохраняет некоторый смысл и в случае функциональных 
языков программирования - скажем, функция \s{sqrt}, вычисляющая квадратный 
корень, действительно очень близка к математическому определению; 
такая функция называется \emph{чистой}. Но существует и отличие. 
В языках программирования возникает понятие \emph{побочного эффекта} ---
любое действие может, помимо 
вычисления требуемого значения, как-то повлиять на остальной мир.

Например, как вы думаете, функция \s{format\_hard\_disk\_drive}, берущая в 
качестве аргумента целое число и возвращающая строчку 
\s{"винчестер отформатирован"} --- действительно ли это только функция из 
множества целых чисел в множество строк? 

Впрочем и другие, менее разрушительные функции, также могут иметь побочный 
эффект. Например, функция печати строки на экране - ценность ее только в 
побочном эффекте. 
Поэтому мы будем придерживаться другого, более утилитарного, взгляда. 
Функция --- это просто фрагмент кода программы. Функция может быть
\emph{вызвана}, то есть ее код может быть исполнен.

\subsection{Определение функций}

Функцию можно \emph{определить} с помощью следующей конструкции:

\begin{bnf}\begin{eqnarray*}
\n{<определение функции>}   &::=&
  \s{let} [\s{rec}] \n{<идентификатор>} \n{параметры} = 
    \n{<выражение>}\s{;;}\\
\n{<параметры>} & ::= & \n{<идентификатор>}^+ | \s{()}
\end{eqnarray*}\end{bnf}

Отличие определения функции от определения константы --- наличие формальных 
параметров, указываемых через пробел после идентификатора функции 
перед знаком равенства. Нетерминал
\n{<идентификатор>} --- это тот же нетерминал, что и в определении констант.

\begin{example}
\begin{verbatim}
let average a b = (a+b)/2;;
\end{verbatim}
\end{example}

Эта функция берет два аргумента и возвращает их полусумму. 
Константы \s{a} и \s{b} называются формальными параметрами --- 
формальными потому, что при вызове функции вместо них подставляются 
фактические параметры:

\begin{verbatim}
print_int (average 3 9);;
\end{verbatim}

Вычисление значения выражения мы можем представить так: вместо букв \s{a} и \s{b}
в описании функции подставляются числа 3 и 9:

\begin{verbatim}
let average 3 9 = (3+9)/2;;
\end{verbatim}
 
Выражение справа от равенства (тело) функции вычисляется и результат (6)
подставляется в место вызова функции:

\begin{verbatim}
print_int (6);;
\end{verbatim}

В тех случаях, когда функция не имеет параметров, мы должны по правилам
языка Окамль указать какой-нибудь фиктивный параметр --- иначе функция
превратится в константу. Обычно такое нужно из-за побочного эффекта
функции, поскольку чистая функция, не зависящая от аргументов --- это
и есть константа.

Существует специальный тип, который используется для указания
фиктивного значения там, где оно нужно - это тип \s{unit}. Он имеет ровно 
одно значение, записываемое как \s{()}.

Для примера покажем, как описать функцию, печатающую приветствие:

\begin{verbatim}
let print_hello () = print_string "Привет!";;
\end{verbatim}

Здесь значение \s{()} исполняет роль формального формального параметра.
Вызов выглядит уже знакомо:

\begin{verbatim}
print_hello ();;
\end{verbatim}

Здесь значение \s{()} исполняет роль формального фактического параметра.

\subsection{Возвращать или печатать?}

Существенным вопросом, касающимся функций, является вопрос о результате
их работы. Предположим, у нас стоит задача написать функцию, вычисляющую
площадь треугольника.

У нас есть 2 варианта:
\begin{verbatim}
let triangle_area1 a h = a *. h /. 2.;;
let triangle_arge2 a h = print_float (a *. h /. 2.);;
\end{verbatim}

Первый вариант функции \emph{возвращает} значение площади. 
Функция \s{triangle\_area1} берет 2 аргумента с плавающей точкой и 
возвращает плавающий результат.
Второй же вариант печатает площадь треугольника на экране --- а в вызвавшее
функцию выражение возвращает значение типа \s{unit}. 

Не смотря на то, что разница между этими вариантами может казаться 
незначительной --- площадь же все равно вычисляется --- тем не менее, 
в чуть более сложных задачах этот вопрос может стать принципиальным.

Дело в том, что значение функции может быть использовано в дальнейших 
вычислениях, и тем самым появляется много дополнительных случаев для 
ее применения. Функция, печатающая вычисленное значение на экране, такой 
возможности лишена. Поэтому везде в задачах, где явно не указано иное, 
предполагается, что от вас требуется написание функции, к которой при 
необходимости прилагается отдельный код, печатающий ее результат:

\begin{verbatim}
let a = read_float ();;
let h = read_float ();;
let triangle_area a h = a *. h /. 2;;
print\_float (triangle_area a h);;
\end{verbatim}

\section{Алгебраические типы данных}

Представим, что мы пишем функцию для вычисления корня линейного
уравнения $$a \cdot x + b = 0$$ 
Если $a \ne 0$, то результат функции --- плавающее число $-\frac{b}{a}$.
Однако, функция может вернуть \emph{не только} плавающее число. 
Например, если $a = 0$ и $b = 0$, то решением уравнения будет любое число,
и никакое конкретное число не будет являться полным ответом.
Также, если $a = 0$ и $b \ne 0$, то решений у уравнения нет вообще.

Таким образом, функция может вернуть один из трёх различных вариантов
ответа: <<один корень>> (в этом случае мы уточняем ответ, указывая
этот корень), <<любое плавающее число --- корень>> и <<корни отсутствуют>>. 

Для таких ситуаций --- а также для многих других, когда у значения есть
несколько вариантов, --- в языке Окамль предусмотрены 
\emph{алгебраические типы}, которым и посвящён этот раздел.

\subsection{Определение алгебраического типа данных}

Давайте определим пользовательский тип данных, способный хранить 
корни линейного уравнения.

Мы уже поняли, что корни уравнения --- это один из трёх вариантов:
<<один корень>> (с указанием этого корня), <<любое плавающее число --- корень>> и <<корни отсутствуют>>. 
В Окамле необходимо называть варианты одним словом, начинающимся с большой буквы: 
пусть они называются \verb!One!, \verb!Any! и \verb!None!. Названия были выбраны
произвольно, на основании английских слов, значащих <<один>>, <<любой>> и
<<никакой>> соответственно --- мы могли выбрать любые другие. 
А теперь составим определение типа:

\begin{verbatim}
type roots = One of float | Any | None;;
\end{verbatim}

Тут задано четыре имени: имя типа (\verb!roots!) и три имени \emph{конструкторов типа} ---
вариантов значений (\verb!One!, \verb!Any!, \verb!None!). 

Пара слов об имени типа. Каждый тип имеет своё имя, мы уже знакомы с некоторыми 
(\verb!int!, \verb!string! и т.п.), и алгебраических типов нам тоже, весьма вероятно,
потребуется не один. Чтобы типы различать, им нужно давать уникальные имена.

Обратите внимание на первый вариант: мы потребовали, чтобы с вариантом \verb!One! 
указывалось ещё и число типа \verb!float!.

%Соответственно, если мы желаем указать значение <<единственный корень, равный 1.4>>,
%то теперь можем написать это так: \verb!One 1.4!, а если требуемое значение --- <<корни отсутствуют>> --- 
%то напишем \verb!None!.

Несколько примеров значений:

\begin{verbatim}
let a = One 1.4;; (* Единственный корень, равный 1.4 *)
let b = Any;;     (* Корень — любое плавающее число *)
let c = None;;    (* Корни отсутствуют *)
\end{verbatim}

Теперь мы можем даже привести функцию, вычисляющую решение линейного 
уравнения, и возвращающую его решение через значение алгебраического типа roots.

\begin{verbatim}
type roots = One of float | Any | None;;

let solve a b c = 
  if a <> 0. then One ((c -. b) /. a)
  else if b = c then Any
  else None;;
\end{verbatim}



%Давайте разберём теперь, как определять алгебраические типы в общем виде:
%
%\begin{bnf}\begin{eqnarray*}
%\n{<описание-алгебраического-типа>} &::=& \s{type\ } \n{<имя>} \s{\ =\ }
%      [\s{|}] \n{<вариант>} \left\{\s{|} \n{<вариант>}\right\}^*{}\\
%\n{<вариант>} &::=& \n{<имя-варианта>} \left[ \s{of} \n{<сигнатура>} \right]\\\\
%\n{<сигнатура>} &::=& \n{<имя-типа>}\\
% &|& \n{<сигнатура>} \s{->{}} \{\n{<сигнатура>}\}^+\\
% &|& \s{(} \n{<сигнатура>} \s{)}
%\end{eqnarray*}\end{bnf}

%Заметьте, для того, чтобы параметризовать варианта, надо указать тип параметра. 
%Неформально с типами мы уже хорошо знакомы, но записывать их \emph{сигнатуры}
%(т.е. обозначения) нам пока не доводилось. 

\subsection{Сопоставление с образцом}

Теперь мы умеем создавать значения алгебраических типов, 
надо научиться их использовать. Например, печатать.

С вариантами без параметров можно справиться и с имеющимися конструкциями,
с использованием условного оператора:
\begin{verbatim}if x = Any then print_string "Любое число"\end{verbatim}%
Но как напечатать корень уравнения, который удалось найти, в случае \s{One}?

Для этого существует сопоставление с образцом.

%Вы должны заметить нечто общее между конструкцией \s{match} и Марковским
%Алгорифмом. Их семантика действительно похожа:
%правила перебираются сверху вниз, и как только найдется подходящее ---
%как только аргумент окажется сопоставим с образцом, указанным в левой
%части правила --- так сразу выражение, стоящее справа, вычисляется, и 
%его результат становится результатом всей конструкции.
%Но собственно процесс сопоставления с образцом отличается. 

%Давайте в этом примере разбираться.

Образец --- это одно из следующего:

\begin{enumerate}
\item литерал, например, \verb!23!, \verb!932.4!, \verb!()!, \verb!true!, \verb!"math"!;
\item имя переменной, например, \verb!x!, \verb!v923!, \verb!_!;
\item кортеж (упорядоченная $n$-ка) других образцов: $(p_1, p_2, \dots, p_n)$ --- образец, если $p_1, \dots p_n$ --- образцы;
например, \verb!(23, x)!, \verb!(23, (x, 17))!;
\item конструктор алгебраического типа без параметра, например \verb!None! или \verb!Any!;
\item конструктор алгебраического типа с параметром: $K p$ --- образец, если $K$ --- конструктор, а $p$ --- образец; например, \verb!One x!.
\end{enumerate}

Сопоставление с образцом в Окамле используется в разных конструкциях,
посмотрим, как это работает в \verb!match!.

\subsection{Конструкция \s{match}}

Рассмотрим, скажем, следующий пример:

\begin{verbatim}
match s with
    0 -> print_string "Ноль"
  | x -> print_string "Не ноль, "; print_int x;;
\end{verbatim}

В конструкции \verb!match! слева от стрелки пишут образцы, а справа --- соответствующие
им выражения.
Первый образец (\verb!0!) сопоставится только, если $s = 0$. 
В остальных случаях сопоставится второй образец (\verb!x!), при этом, как и в
конструкции \verb!let!, переменная \verb!x! получит нужное для сопоставления значение.

Теперь мы можем легко написать функцию, печатающую значение типа roots:

\begin{verbatim}
let print_roots v =
  match v with
    Any    -> print_string "Любое число"
  | None   -> print_string "Корней нет"
  | One f  -> print_float f;;
\end{verbatim}

Существует специальный идентификатор для констант, значения которых нас не 
интересуют --- символ подчеркивания (\s{\_}). При сопоставлении это --- 
обычная константа и, например, сопоставление с образцом \s{One \_} 
происходит так же, как и с образцом \s{One f}. Но в случае удачного 
сопоставления значение константе s{\_} не присваивается. Наиболее типичный 
пример использования этой константы выглядит так:

\begin{verbatim}
match v with
  One f -> print_float f              (* печатаем корень *)
| _ -> print_string "Одного корня нет";; (* а здесь все остальные, 
			неинтересные нам случаи *)
\end{verbatim}

В приведенном выше примере мы печатали результат и значение функции нас
не интересовало. Но все может быть и иначе:

\begin{verbatim}
let finite_roots v = (* функция возвращает true, если ее аргумент -
			не более чем конечное количество корней *)
  match v with
    One _ -> true
  | None  -> true
  | Any   -> false;;
\end{verbatim}


\section{Списки}

\subsection{Определение}

Слово список всем хорошо знакомо: список дел, список класса, список оценок.
В программировании так называют структуру данных, позволяющую хранить 
упорядоченный набор значений, при этом важно, что мы умеем быстро добавлять
элемент в начало списка и быстро убирать начальный элемент. 

Определение, данное выше, весьма размытое, но с помощью Окамля мы можем его 
формализовать.

Списком назовём алгебраический тип, заданный следующим образом:
\begin{verbatim}
type 'a list = Nil | Cons of 'a * ('a list)
\end{verbatim}
Напомним, что значит эта запись: список есть либо константа \verb!Nil!, 
либо конструктор типа \verb!Cons!, имеющий аргументом пару из значения и 
списка.

Поясним, почему \verb!Cons (3, Cons (5, Nil))! --- список. Начнём с
\verb!Nil!, который является списком по первому правилу из определения.
Раз так, значит, \verb!Cons (5, Nil)! --- тоже список,
поскольку у \verb!Cons! аргументом как раз и является пара из значения 
(\verb!5!) и списка (\verb!Nil!). А раз \verb!Cons (5, Nil)! --- список,
то и всё выражение --- тоже список.

Поскольку списки --- одни из основных конструкций Окамля, то для них
предопределён особый синтаксис: вместо \verb!Nil! используются 
квадратные скобки (\verb![]!),
а вместо \verb!Cons! --- двойное двоеточие (\verb!::!). Таким образом, 
\verb!Cons (3, Cons (5, Nil))! записывается как \verb!3 :: (5 :: [])!.

Также, список можно задавать явным перечислением значений в квадратных
скобках через точку с запятой: \verb![3;5]!.

Ещё определим несколько полезных слов. \emph{Пустой} список --- список, 
состоящий только из \verb![]!, т.е. список, не содержащий никаких значений.
В противоположность ему, любой непустой список 
$a_1 \verb!::! (a_2 \verb!::! \dots \verb!::! (a_n \verb!::! \verb![]!))$
можно разбить на \emph{голову} и \emph{хвост}: голова --- это $a_1$, а хвост --- это
остающийся после удаления головы список $a_2 \verb!::! \dots \verb!::! (a_n \verb!::! \verb![]!)$.
Или иначе: если рассмотреть конструктор типа \verb!::!, то левый его аргумент ---
это голова, а правый --- хвост (\emph{голова} \verb!::! \emph{хвост}).

\subsection{Доступ к элементам списка}

Как мы уже проходили, доступ к элементам алгебраического типа в первую очередь
осуществляется с помощью сопоставления с образцом и конструкции \verb!match!, 
и списки --- не исключение.

Классический, самый прямой, вариант применения этой конструкции покажет функция 
вычисления длины списка:

\begin{verbatim}
let rec length l = match l with
    [] -> 0
  | h::t -> 1 + length t;;
\end{verbatim}

Два конструктора в определении списка --- значит, два варианта в \verb!match!.
Первый случай --- когда список пуст, второй --- когда список собран из некоторых
головы и . Самой по себе головой мы не интересуемся, ведь, чтобы посчитать
количество элементов в списке, вникать в сами эти элементы не обязательно.
Поэтому значение \verb!h! не указано справа от стрелки, только слева --- а справа
мы всегда прибавляем 1 к длине хвоста. 

В самом деле: пусть есть непустой список \verb!l!. Раз он непустой, то он образован
конструктором <<двойное двоеточие>> из какого-то элемента \verb!h! и какого-то ещё
списка \verb!t!: \verb!l = h::t!. Понятно, что длина \verb!t! на один элемент
меньше, чем длина всего списка. 

\subsection{Пример: красивая печать списка}

Всё, что дальше будет излагаться про списки, ничего принципиально нового не содержит:
это будут примеры на использование уже рассказанных выше конструкций. Однако, 
данные примеры могут быть поучительны.

В этом подразделе рассмотрим задачу печати список целых чисел. Мы можем сделать это
в целом следуя идее, использованной при вычислении длины.

\begin{verbatim}
let rec print_int_list l = match l with
    [] -> ()
  | h::t -> print_int h; print_int_list t;;
\end{verbatim}

Отличия от функции \verb!length! здесь только в том, что мы не возвращаем длину
списка, а печатаем его --- возвращаем же мы \verb!()!. Напомним, что данное 
значение принадлежит типу \verb!unit! и используется там, где формально указать
значение нужно, а фактически написать нам нечего --- ведь весь смысл данной
функции напечатать текст на экране; новых значений с её помощью мы не создаём.

Итак --- в случае пустого списка мы возвращаем <<ничего>> --- то есть \verb!()!, а в случае
наличия головы и хвоста печатаем голову, после чего рекурсивно вызываем печать
для хвоста.

Однако, вызов \verb!print_int_list [1;2;3]! выдаст нам \verb!123! --- напечатает
числа без пробелов. Это можно исправить, добавив разделителей между числами, например, так:

\begin{verbatim}
let rec print_int_list l = match l with
    [] -> ()
  | h::t -> print_int h; print_string ","; print_int_list t;;
\end{verbatim}

Однако, результат не станет идеальным: мы получим \verb!1,2,3,!. Каша из цифр
исчезла, но появилась заключительная запятая, которая будет сбивать с толку.
Причина её появления очевидна: мы \emph{всегда} печатаем запятую после числа,
однако, в красивой записи запятая разделяет два числа, и за последним числом 
не нужна. Поэтому нам нужно две версии печати числа: с запятой в конце или без неё.

Идея, как мы могли бы отличить эти два случая, проста --- у последнего двойного двоеточия
списка пустой хвост: $a_1 \verb!::! (a_2 \verb!::! \dots \verb!::! (a_n \verb!::! \verb![]!))$.
Иными словами, когда последнее число списка $a_n$ окажется в переменной \verb!h! образца,
в переменной \verb!t! окажется пустой список. Это наблюдение приводит к следующей 
программе:

\begin{verbatim}
let rec print_int_list l = match l with
    [] -> ()
  | h::[] -> print_int h
  | h::t -> print_int h; print_string ","; print_int_list t;;
\end{verbatim}

Обратите внимание, что мы вставили вариант печати последнего элемента списка перед 
общим случаем. Конструкция \verb!match! перебирает варианты строго сверху вниз,
и выбирает первый подходящий образец. Если бы мы поставили обработку последнего
элемента ниже общего случая, соответствующий код никогда бы не вызывался.

Последний штрих: данная функция печатает только целочисленные списки.
А что делать, если нам нужно печатать списки других типов, например, состоящие из строк?
Окамль в общем случае не знает, как печатать значение произовольного типа. Для этой цели
давайте функции печати списка передавать функцию печати элемента:

\begin{verbatim}
let rec print_list f l = match l with
    [] -> ()
  | h::[] -> f h
  | h::t -> f h; print_string ","; print_list f t;;
\end{verbatim}

Вызывать теперь её нужно, правда, чуть хитрее:

\begin{verbatim}
print_list print_int [1;2;3]
\end{verbatim}

Функция \verb!print_list! теперь берёт на вход два параметра, и первый --- это функция
\verb!print_int!, которая печатает элементы. Если нам в другой ситуации потребуется
печать списка строк, мы без труда сделаем это: \verb!print_list print_string ["a";"b";"c"]!.

\subsection{Пример: соединение двух списков}

Определим функцию \verb!append!, соединяющую два списка --- скажем, чтобы из списков 
\verb![1;2]! и \verb![3;4]! получался бы список \verb![1;2;3;4]!. Главная сложность здесь ---
увидеть, как организовать рекурсию.

Возьмём первый список и разберём возможные случаи. Если первый список пуст, 
результатом функции, очевидно, будет второй: добавление пустого списка не изменяет 
результат. Если же первый список содержит голову $a_1$ и хвост $[a_2; \dots; a_n]$, 
мы можем рекурсивно соединить хвост со вторым списком, а затем добавить голову 
к результату: 
$$append [a_1; a_2; \dots a_n] [b_1; b_2; \dots b_n] = append (a_1 :: [a_2; \dots a_n]) [b_1; \dots; b_n] = $$
$$a_1 :: append [a_2; \dots a_n] [b_1; b_2; \dots b_n]$$
Ведь всё равно элемент $a_1$ должен быть первым в итоговом списке --- поэтому
мы его можем сразу от первого списка отцепить, и прицепить уже к результату 
рекурсивного вызова.

Это даёт нам следующую программу:

\begin{verbatim}
let rec append a b = match a with
    [] -> b
  | h::t -> h::(append t b)
\end{verbatim}

Поскольку данная операция очень часто нужна при работе со списками, она
встроена в язык и у неё специальное имя --- символ <<коммерческое „эт“>>
(\verb!@!), он же неформально называется <<собака>>. 
Соответственно, следующее странное на первый взгляд равенство в Окамле 
вполне осмысленно: 
\verb![1;2] @ [3;4] = [1;2;3;4]!

\subsection{Пример: разворот списка}

И следующий пример, который мы разберём --- функция разворота, делающая из списка $[a_1;a_2;\dots;a_n]$
список $[a_n;a_{n-1};\dots;a_2;a_1]$.

Прямолинейное решение, \verb!match! по аргументу, вполне сработает.
Для пустого списка его разворот также пуст, для непустого --- развернём хвост, а голову списка
присоединим к концу:

\begin{verbatim}
let rec reverse l = match l with
    [] -> []
  | h::t -> reverse t @ [h]
\end{verbatim}

Сразу обращаем внимание, что код \verb!h::t -> reverse t :: h! был бы 
некорректен. Напомним, что двойное двоеточие \emph{всегда} требует слева элемент (голову),
справа список (хвост). В данном же случае получается, что слева указан хвост, а справа --- голова.
Это неизбежно приведёт к ошибке компиляции.

Также, возможно, кого-то озадачит код \verb![h]! --- но это всего лишь список, содержащий
\verb!h! в качестве единственного своего элемента. Для Окамля нет никакой разницы между
записями \verb![5]!, \verb![1+1*4]! или даже \verb!let x = 5 in [x]!, внутри квадратных 
скобок может стоять любое выражение.

Однако, можно предложить и другое решение:

\begin{verbatim}
let rec revh l a = match l with
    [] -> a
  | h::t -> revh t (h::a);;

let reverse l = revh l [];;
\end{verbatim}

Проследим за тем, как данный код развернёт трёхэлементный список \verb![2;3;4]!
(мы напишем только цепочку равенств, более подробный анализ исполнения кода оставим
читателю):

\begin{verbatim}
revh [2;3;4] [] = revh [3;4] [2] = revh [4] [3;2] = revh [] [4;3;2] = [4;3;2]
\end{verbatim}

Функция \verb!rehv! на каждом рекурсивном вызове перекидывает
по элементу из исходного списка в итоговый, пока не развернёт его целиком. 

\section{Кортежи (упорядоченные $n$-ки)}

Бывают ситуации, когда нужно запомнить (передать) два числа одновременно.

Давайте напишем функцию, которая возвращает все целые числа, отличающиеся от 
данного целого числа на $1$. 
Скажем, для $4$ функция должна вернуть два числа: $3$ и $5$.
Аналогично, для $-18$ --- числа $-17$ и $-19$.

Содержательно задача тривиальная (по числу $x$ вернуть $x+1$ и $x-1$),
но пока что мы писали функции, возвращающие не более одного числа.
Мы можем написать две функции: одна будет прибавлять 1, а другая --- вычитать,
но есть другой способ.
, называемый кортежем или
упорядоченной $n$-кой.

Сперва научимся создавать значения этого типа. Для этого нужно перечислить
значения, которые вы хотите положить в кортеж, через запятую и окружить 
скобками:

Примеры:
\begin{verbatim}
let a = (1,2);;
let b = (1,3,"asdf",9.);;
let c = ((1,2),(3,4));;
\end{verbatim}
Заметим, что константа c хранит кортеж из двух других кортежей. Это, 
естественно, вполне допустимо.

Кортежи также можно указывать в сопоставлении с образцом:

\begin{verbatim}
type roots = One of float | Any | None

let roots_info t =
  match t with
    (Any, _) -> "бесконечное количество корней"
  | (_, Any) -> "бесконечное количество корней"
  | _        -> "конечное количество корней";;

print_string (roots_info (One 1., Any));;
\end{verbatim}

И, конечно, кортежи можно указывать как параметры в алгебраических типах:

\begin{verbatim}
type roots2 = Two of float*float | One of float | Any | None
\end{verbatim}

\subsection{Формальная грамматика}

\begin{bnf}\begin{eqnarray*}
\n{<сигнатура-кортежа>} &::=& \n{<сигнатура-типа>} 
  \left\{ \s{*} \n{<сигнатура-типа>} \right\}^+ \\
\n{<литерал-кортежа>}   &::=& 
  \s{(} \n{<выражение>} \left\{\s{,} \n{<выражение>} \right\}^+ \s{)}\\
\n{<образец-кортежа>}   &::=& \s{(} \n{<образец>} \left\{ \s{,} \n{<образец>}\right\}^+ \s{)}
\end{eqnarray*}\end{bnf}

\section{Рекурсия}

Что можно сказать про такой тип?
type a = S of a | Nil;;

Отличие его от обычного алгебраического типа - использование своего имени 
внутри описания. Выглядит, возможно, довольно непривычно для вас, но если
будем действовать по порядку, все получится. Воспользуемся нашим упрощенным
определением (тип - это множество значений), и попробуем найти значения,
которые этому типу соответствуют.

\begin{itemize}
\item Давайте сперва посмотрим на конструктор \verb!Nil!. 
У \verb!Nil! нет параметров, поэтому значение \verb!Nil!, очевидно, принадлежит 
типу \verb!a!.
\item Теперь будем разбираться с конструктором \verb!S!. Этот конструкторв имеет 
параметр типа \verb!a!, значит, если \verb!x! - это значение типа \verb!a!, 
то \verb!S(x)! - это тоже значение типа \verb!a!.

\item Но раз так, то и \verb!S(S(Nil)! -- типа \verb!a!.
И вообще, любое выражение вида \verb!S(S(...S(Nil)...))! -- типа \verb!a!. 
Но, поскольку кроме рассмотренных конструкторов \verb!S! и \verb!Nil! 
никаких других способов 
образовать тип a в его описании не указано, у него нет и других значений. 
Таким образом, мы исчерпывающе описали тип a.
\end{itemize}

Теперь разберемся, как нам, например, напечатать значение этого типа.
Первая идея, приходящая в голову, не сработает:

\begin{verbatim}
let print_a v = 
  match v with
    Nil       -> print_string "Nil"
  | S(Nil)    -> print_string "S(Nil)"
  | S(S(Nil)) -> print_string "S(S(Nil))"
  ...
\end{verbatim}
поскольку значений типа бесконечно много.

Поэтому нам потребуется новое понятие - рекурсивная функция. Как видно из
названия, это функция, которая вызывает сама себя. Вот как будет выглядеть
рекурсивная функция печати этого типа:

\begin{verbatim}
let rec string_of_a v =
  match v with
    Nil  -> "Nil"
  | S(x) -> "S(" ^ string_of_a x ^ ")";;

let print_a v = print_string (string_of_a v);;
\end{verbatim}

Обратим внимание на две тонкости в описании \s{string\_of\_a}. 

Во-первых, после ключевого слова \s{let} появилось ранее не встечавшееся ключевое 
слово \s{rec}. Это слово необходимо писать при описании всякой рекурсивной функции. 
Во-вторых, в четвертой строке функции \s{string\_of\_a} происходит ее 
рекурсивный вызов --- вызов описываемой функции.

Возможно, вам непонятно, что значит, что функция вызывает себя саму. 
Действительно, внешне это немного напоминает барона Мюнгхаузена, 
вытаскивающего себя за волосы из болота. 
Поступим в точности как со значениями --- будем разбираться по отдельности.

Для удобства повторим код еще раз:

\begin{verbatim}
let rec string_of_a v =
  match v with
    Nil  -> "Nil"
  | S(x) -> "S(" ^ string_of_a x ^ ")";;
\end{verbatim}

\begin{enumerate}

\item
Чему равно \s{string\_of\_a\ Nil}? Ответ: \s{"Nil"}, поскольку эта
ветвь вычислений не использует рекурсии. Давайте запомним это и более не 
будем думать об этом случае. Для четкости отразим это в таблице:

\begin{tabular}{ll}\\
\hline
Выражение&Результат вычисления\\
\hline
\s{string\_of\_a\ Nil}&\s{"Nil"}\\
\hline\\
\end{tabular}

\item

Чему равно \s{string\_of\_a (S(Nil))}? Проследив выполнение конструкции
\s{match}, видим, что это будет 
\s{"S(" \^\ string\_of\_a x \^\ ")"}. 

Мы столкнулись с рекурсивным вызовом,
но из таблицы мы ведь знаем результат конкретно этого рекурсивного вызова,
ведь \s{(string\_of\_a Nil)} - это \s{"Nil"}.
Поэтому мы можем этот результат подставить:
\s{"S(" \^\ "Nil" \^\ ")"}, то есть \s{"S(Nil)"}.
Отразим и этот факт
в таблице:

\begin{tabular}{ll}\\
\hline
Выражение&Результат вычисления\\
\hline
\s{string\_of\_a\ Nil}&\s{"Nil"}\\
\s{string\_of\_a\ (S(Nil))}&\s{"S(Nil)"}\\
\hline\\
\end{tabular}

\item
Чему равно \s{string\_of\_a\ (S(S(Nil)))}? 
Повторим рассуждение: это 
\s{"S(" \^\ string\_of\_a\ (S(Nil)) \^\ ")"}, то есть
\s{"S(S(Nil))"}, что мы снова можем запомнить:

\begin{tabular}{ll}\\
\hline
Выражение&Результат вычисления\\
\hline
\s{string\_of\_a\ Nil}&\s{"Nil"}\\
\s{string\_of\_a\ (S(Nil))}&\s{"S(Nil)"}\\
\s{string\_of\_a\ (S(S(Nil)))}&\s{"S(S(Nil))"}\\
\hline
\end{tabular}
\end{enumerate}

Легко видеть, что так мы разберем все возможные случаи. Ведь каждое 
значение типа \s{a} имеет конечное число вложенных значений \s{S}, после 
которых, в самой глубине, хранится значение \s{Nil}. Наше же рассуждение, 
идя в обратную сторону, неизбежно дойдет до значения с любым конечным 
количеством букв \s{S}.

Конечно, эти рассуждения не полностью отражают реальный процесс вычислений, 
компьютер не запоминает результат функций, но для первого знакомства с 
рекурсивными функциями такого понимания должно быть вполне достаточно.

\subsection{Натуральные числа}

Напомним, как в аксиоматике Пеано представляются натуральные числа:
постулируется существование константы $0$ и операции прибавления $1$ 
(\emph{инкремента}). 
В формулах мы будем обозначать инкремент штрихами:
число $a''$ --- это число $a$, увеличенное на 1 два раза (то есть $a + 2$), 
а число $3$ представимо как $0'''$. 
Что интересно, этих простых операций достаточно, чтобы определить все 
операции целочисленной арифметики (сложение, вычитание, 
умножение).

\begin{tabular}{ll}\\
\hline
Операция&Определение\\
\hline
Инктремент&$\n{inc} (a) = a'$\\
Сложение&$\n{plus} (0,b) = b$; $\n{plus} (a',b) = (\n{plus} (a,b))'$\\
Декремент&$\n{dec} (a') = a$; $\n{dec} (0) = 0$\\
Вычитание&$\n{minus} (a',b') = \n{minus} (a,b)$; $\n{minus} (0,a) = 0$; $\n{minus} (a,0) = a$\\
Умножение&$\n{mul} (a',b) = \n{plus} (b, \n{mul} (a,b))$; $\n{mul} (0,b) = 0$\\
\hline
\end{tabular}

\begin{example}
Давайте вычислим $3 \cdot 2$ с помощью этих определений. 
Заметим, что 3 --- это $0'''$, а 2 --- это $0''$. Тогда вычисление
$\n{mul} (0''', 0'')$ будет состоять из следующих преобразований:

\begin{tabular}{ll}\\
\hline
Выражение в арифметике Пеано&Обычная запись\\
\hline
  $\n{mul} (0''', 0'')$ = & $3 \cdot 2$\\
  $\n{plus} (0'', \n{mul} (0'', 0''))$ = & $2 + (2\cdot 2)$\\
  $\n{plus} (0'', \n{plus} (0'', \n{mul} (0', 0'')))$ = & $2 + (2 + (1\cdot 2))$\\
  $\n{plus} (0'', \n{plus} (0'', \n{plus} (0'', \n{mul} (0, 0''))))$ = & $2 + (2 + (2 + (0\cdot 2)))$\\
  $\n{plus} (0'', \n{plus} (0'', \n{plus} (0'', 0)))$ = & $2 + (2 + (2 + 0))$\\
  $\n{plus} (0'', \n{plus} (0'', (\n{plus} (0', 0))'))$ = & $2 + (2 + ((1 + 0) + 1)$\\
  $\n{plus} (0'', \n{plus} (0'', (\n{plus} (0, 0))''))$ = & $2 + (2 + ((0 + 0)) + 2)$\\
  $\n{plus} (0'', \n{plus} (0'', 0''))$ = & $2 + (2 + 2)$\\
  $\n{plus} (0'', (\n{plus} (0', 0''))')$ = & $2 + ((1 + 2) + 1)$\\
  $\n{plus} (0'', (\n{plus} (0, 0''))'')$ = & $2 + ((0 + 2) + 2)$\\
  $\n{plus} (0'', 0'''')$ = & $2 + 4$\\
  $(\n{plus} (0', 0''''))'$ = & $(1 + 4) + 1$\\
  $(\n{plus} (0, 0''''))''$ = & $(0 + 4) + 2$\\
  $0''''''$ & $6$\\
\hline\\
\end{tabular}
\end{example}

Математическую сторону этих определений (доказательство корректности и т.п.)
мы опустим, а вот на программистскую как раз и обратим внимание.
Ведь тип данных \s{a}, описанный выше, и является как раз типом данных,
представляющим натуральные числа <<в стиле аксиоматики Пеано>>.

Покажем, например, как реализовать операцию инкремента, это очень просто:
\begin{verbatim}
let inc x = S x;; (* дописываем одну букву S - прибавляем один *)
\end{verbatim}

Чуть посложнее операция сложения, там в определении есть два случая.
Первый случай --- когда первое слагаемое равно 0, и второй случай, когда
оно содержит применение как минимум одной операции прибавления 1 ($'$). 
Впрочем, и здесь мы можем впрямую следовать определению.
\begin{verbatim}
let rec plus a b = 
  match a with
    Nil -> b
  | S a -> S (plus a b);;
\end{verbatim}

А так реализуется вычитание:
\begin{verbatim}
let rec minus a1 b1 = 
  match (a1, b1) with  (* разбор случаев из определения *)
    (S a, S b) -> minus a b
  | (Nil, a) -> Nil
  | (a, Nil) -> a;;
\end{verbatim}

И завершит этот пример функция преобразования значений типа \s{a} в 
значения типа \s{int}:

\begin{verbatim}
let rec int_of_a v =
  match v with
    Nil -> 0
  | S(x) -> 1 + int_of_a x;;
\end{verbatim}

Здесь мы действуем впрямую по определению: если значение типа \s{a}
имеет вид \s{S(x)} --- прибавление 1 к $x$ --- то надо вычислить значение
для \s{x}, а затем прибавить к нему 1.

\subsection{Списки}

Мы уже познакомились с одним частным случаем списка --- типом \s{a} из 
предыдущего
параграфа. Теперь настала пора познакомиться с ними в общем случае.

Вообще, список можно было бы описать примерно так:
\begin{verbatim}
type 'a list = Cons of 'a * 'a list | Nil
\end{verbatim}

Как видите, это тип похож на тип \s{a}, главное его отличие --- этом
непонятном символе \s{'a} перед именем \s{list}, а также в наличии 
дополнительного значения в каждом элементе \s{Cons}.
Естественно, значения тоже похожи:
\begin{verbatim}
Cons (1, Cons (2, Cons (3, Nil)))
\end{verbatim}

В силу частоты использования, тип список встроен в язык,
и конструкторы типа \s{Cons} и \s{Nil} имеют специальные
имена: \s{::} и \s{[]} соответственно, а вместо 
\s{Cons (1, Cons (2, Cons (3, Nil)))}
надо писать \s{1 :: 2 :: 3 :: []}
Также, есть еще один вариант записи этого списка:
\s{[1;2;3]}, 

Естественно, эти сокращения так же применяются и в сопоставлении с образцом.

Важно! Конструктор типа \s{::} несимметричен. Слева от него всегда должен
быть указан элемент, а справа --- список, например так: \s{1 :: []}. 
Выражение \s{[] :: 1} некорректно.

Но при этом \s{[1]::[[1]]}, так же, как и \s{[1]::[]} --- корректные списки,
состоящие из списков целых чисел.

Пример (подсчет длины списка, то есть количества \verb!::!):

\begin{verbatim}
let rec list_length l =
  match l with
    [] -> 0
  | l1::ls -> 1 + list_length ls;;
\end{verbatim}

В том случае, если список состоит только из \s{[]}, он имеет длину 0.
Если список - это \s{l1::[]}, то длина равна $1 + \s{list\_length\ []}$, то есть 1.
Если список - это \s{l2::l1::[]}, то длина - $1 + \s{list\_length\ l1::[]}$, то есть 2.
И в общем случае, если есть список \s{l1::ls}, то его длина равна
$1 + \s{list\_length\ ls}$.

Список может хранить элементы любого типа, но этот тип должен быть
для всех элементов один и тот же.
Тип списка \s{[1;2;3]} --- \s{int list}, тип списка \s{["1";"a";"c"]} --- 
\s{string list},
список \s{[[1];[2];[]]} имеет тип \s{int list list} (для ясности можно поставить
скобки: \s{(int list) list}), но списка \s{[1;"2"]} быть не может.

%<сигнатура-типа-списка> ::= <сигнатура-типа> list
%<конструктор-cons-для-списка> ::= <выражение> :: <выражение>
%<конструктор-nil-для-списка> ::= []
%<литерал-списка> ::= [ [<выражение> (; <выражение>)*] ]



\subsection{Сигнатура типа}

\emph{Сигнатурой} типа мы будем называть его обозначение на языке Окамль.
Сигнатуры элементарного типа нам уже знакомы:

\begin{tabular}{ll}
\hline
Тип&Сигнатура типа\\
\hline
строчки            &string\\
целые числа        &int\\
плавающие числа    &float\\
булевские (логические) значения &bool\\
одноэлементный тип &unit\\
\hline
\end{tabular}

Но мы уже умеем строить значения не только этих простых типов, но и более 
сложных конструкций: речь идет о функциях.

Рассмотрим следующую функцию:
\begin{verbatim}
let x a b = if a > 0 then b else 1.;;
\end{verbatim}

Эта функция берет два аргумента (целочисленный и плавающий) и возвращает 
плавающее число. Соответственно, сигнатура этой функции имеет следующий вид:
\begin{verbatim}
x: int -> float -> float
\end{verbatim}
Первая из стрелок \s{->}, указанная в сигнатуре, разделяет сигнатуры типов
аргументов функции, а вторая отделяет тип аргумента от типа результата функции.

Как можно догадаться, какой тип у функции?
Когда мы пишем функцию, мы уже должны иметь представление, аргументы каких 
типов ей передаются и какой тип значений она должна возвращать.
Ведь мы же знаем, что она должна делать.
Тем не менее компилятор не знает о наших предположениях, он судит по 
тому коду, который мы написали --- и, основываясь на нем, автоматически
выводит типы аргументов и результата. Попробуем сделать это и мы.

Это как головоломка. Вот есть такой код:
\begin{verbatim}
let z x y z = if (x + y) / 2 > 0 then "false" else z;;
\end{verbatim}
Что можно по нему заключить?
\begin{enumerate}
\item Формальные параметры \s{x} и \s{y} складываются между собой
операцией \s{+}, которая требует, чтобы левый и правый операнды
имели тип \s{int}. Значит, \s{x} и \s{y} могут иметь только тип \s{int}
--- иначе код содержит ошибку.
\item Тип результата --- \s{string}, поскольку результат вычисления 
операции \s{if} является результатом всей функции --- а одна из ветвей
возвращает значение \s{"false"}, имеющего тип \s{string}.
\item И наконец ветви операции \s{if} должны иметь одинаковый тип ---
значит, тип результата совпадает с типом параметра \s{z}.
\item Итого, аргументы \s{x} и \s{y} типа \s{int}, аргумент \s{z} типа
\s{string}, и результат функции --- тоже типа \s{string}.
Значит, сигнатура типа функции \s{z} такова: 
\s{int\ ->\ int\ ->\ string\ ->\ string}.
\end{enumerate}

Рассмотрим еще несколько примеров:

\begin{tabular}{ll}\\
\hline
Значение&Сигнатура типа\\
\hline
\s{let v () = "это строка";;}                      &\s{v: unit -> string}\\
\s{let w () = print\_string "Привет\n";;}&\s{w: unit -> unit}\\
\s{let x a b = (a = 0) || (b = 0.);;}              &\s{x: int -> float -> bool}\\
\s{let y a b c = (a = "{}"{}) || (b = 0) || c;;}       &\s{y: string -> int -> bool -> bool}\\
\s{let z x = if x = "{}4"{} then 1 else -1;;}          &\s{z: string -> int}\\
\hline\\
\end{tabular}



\end{document}
