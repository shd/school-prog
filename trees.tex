\documentclass[12pt,a4paper,oneside]{book}
\usepackage[utf8]{inputenc}
\usepackage[english,russian]{babel}
\usepackage[left=2cm,right=2cm,top=2cm,bottom=2cm,bindingoffset=0cm]{geometry}
\begin{document}

Пожалуй, мы можем предложить два различных способа хранить
значения: в узлах дерева и в листьях.

\subsubsection{Хранение значений в листьях дерева}

В качестве примера из реального мира возьмём почтовые адреса:
они явно образуют дерево, в котором значение сопоставляется листьям, 
узлы же выполняют в основном техническую функцию.

Рассмотрим адрес: Санкт-Петербург, ул. Егорова, дом 24, квартира 77.
Здесь ни по какому префиксу нельзя доставить письмо: ни <<Санкт-Петербург>>, 
ни <<Санкт-Петербург, ул. Егорова>>, ни даже <<Санкт-Петербург, ул. Егорова, дом 24>> 
не соответствуют какому-то получателю, и письмо можно доставить только
имея полный путь от корня до листа.

Двоичные деревья со значениями только в листьях могут быть заданы так:

\begin{verbatim}
type ('k,'v) tree = Node of 'k * ('k,'v) tree * ('k,'v) tree 
                  | Leaf of 'k * 'v
\end{verbatim}

\subsubsection{Хранение значений в узлах дерева}

Другая ситуация возникает, когда листья не выполняют какой-то особой функции --- 
как в случае пригородной железнодорожной линии. Конечно, остановки могут отличаться 
по путевому развитию, частоте движения поездов, загруженности пассажирами,
но до любой из станций на линии можно купить билет.

Более того, для упрощения мы можем считать все остановки стоящими в узлах,
а листья считать незначащими. Ведь и в жизни линия часто физически продолжается 
за конечной платформой, куда электричка, высадив всех пассажиров, 
отправляется для смены направления.

Двоичные деревья со значениями в узлах (в варианте с незначащими листьями) 
могут быть заданы так:

\begin{verbatim}
type ('k,'v) tree = Node of 'k * ('k,'v) tree * ('k,'v) tree 
                  | Leaf
\end{verbatim}

\end{document}