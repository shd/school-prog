\documentclass[12pt,a4paper,oneside]{article}
\usepackage[utf8]{inputenc}
\usepackage[english,russian]{babel}
\usepackage{amsthm}
\usepackage{amssymb}
\usepackage{enumerate}
\usepackage{cmll}
\usepackage[left=1.5cm,right=1.5cm,top=1.5cm,bottom=2cm,bindingoffset=0cm]{geometry}
\usepackage{bnf}
\newcommand{\s}[1]{\texttt{#1}}
\usepackage{proof}

\begin{document}
\theoremstyle{definition}
\newtheorem{definition}{Определение}[section]

\newtheorem{lemma}{Лемма}[section]
\newtheorem{theorem}{Теорема}[section]

\section{Перестановки}

Напомним классическое определение. Перестановка --- биективная функция, отображающая 
множество $X$ на себя.
Однако, в данной главе в качестве множества $X$ мы будем рассматривать только 
множества целых чисел от $0$ до $n-1$ (часто такое множество записывают как
$\overline{0,n-1}$). Соответственно, определение становится таким:

\begin{definition}
Если $\pi: \overline{0,n-1} \rightarrow \overline{0,n-1}$ является биективной,
то она --- \emph{перестановка}.
\end{definition}

\noindent Обозначим множество всех перестановок множества $\overline{0,n-1}$ за $S_n$.

Традиционный способ задания перестановок --- с помощью <<двухэтажной>> записи 
(обычно называемой \emph{подстановкой}):

$$
\left(\begin{array}{ccccc}
0 & 1 & 2 & \dots & n-1\\
\pi(0) & \pi(1) & \pi(2) & \dots & \pi(n-1)
\end{array}\right)
$$

Однако, верхняя строчка в перестановках множества $\overline{0,n-1}$ всегда одинакова, поэтому мы будем её опускать,
и записывать только нижнюю часть: $[\pi(0);\pi(1);\dots;\pi(n-1)]$. Такая запись не является
канонической для математики, но зато она очень удобна для наших целей.
Например, перестановка пяти элементов, меняющая местами 0 и 1 (и оставляющая на месте 2, 3 и 4), 
будет задаваться нами так: $[1;0;2;3;4]$

Также введём специальное обозначение для количества переставляемых элементов.

\begin{definition}
Пусть $\pi: \overline{0,n-1} \rightarrow \overline{0,n-1}$, \emph{мощностью}
перестановки $\pi$ будем называть мощность множества отправления (прибытия).
Будем обозначать эту мощность как $|\pi|$.
Например, $|[0;2;1]| = 3$. 
\end{definition}

Также давайте введём понятие \emph{префикса} перестановки: 

\begin{definition}
Пусть $[a_0; a_1; \dots; a_{n-1}]$ --- некоторая перестановка.
Тогда последовательность чисел $[a_0; a_1; \dots; a_{k-1}]$ при $k \le n$~--- 
\emph{префикс} (начало) перестановки. В частности, префиксами
являются пустая последовательность чисел $[]$ и вся исходная перестановка.
\end{definition}

Легко заметить, что если последовательность чисел $[a_0; a_1; \dots; a_{k-1}]$~---
последовательность попарно различных чисел из диапазона от $0$ до $n-1$:
$$0 \le k \le n, \quad 0 \le a_i < n, \quad a_p \ne a_q \mbox{ при } q \ne q$$
то такая последовательность
может быть продолжена до полной перестановки $(n-k)!$ способами.

\subsection{Лексикографический порядок}

Лексикографический порядок ещё называют словарным. Правила тут должны быть
хорошо известны: чтобы сравнить два слова, нужно сравнить у них первые различающиеся 
буквы в одной и той же позиции. 
Например, слово \emph{РУКА} идёт раньше в словаре, чем слово \emph{РЫБА}, поскольку во второй
позиции первого слова --- буква \emph{У}, а у второго слова --- буква \emph{Ы}, идущая позже.

\begin{definition}
Обозначим за $|S|$ длину слова $S$, и за $S_i$ --- символ строки номер $i$. Нумерацию по
традиции будем вести с $0$.
Например, если $S = \texttt{абрикос}$, то $|S|=7$ и $S_4=\texttt{к}$.
\end{definition}

\begin{definition}
Пусть даны два слова ($S$ и $T$) одинаковой длины.
Слово $S$ меньше слова $T$ (обозначим это отношение за $S \prec T$), 
если существует индекс $i$, что:
$$S_0 = T_0, \quad S_1 = T_1, \quad \dots, \quad S_{i-1} = T_{i-1}, \quad \mbox{но } S_i < T_i$$
\end{definition}

\noindent Данное условие можно обобщить и на случай слов разной длины.
\begin{definition}
$S \prec T$, если выполнено одно из двух условий:
\begin{enumerate}
\item Либо существует индекс $i$, такой, что $0 \le i < \min(|S|,|T|)$ и $S_i < T_i$, но для всех индексов $j$, $0 \le j < i$, выполнено $S_j = T_j$,
\item либо $|S| < |T|$, и для всех индексов $i$ ($0 \le i < |S|$), выполнено $S_i=T_i$
\end{enumerate}
\end{definition}

\noindent Мы легко могли бы написать функцию сравнения двух списков:

\begin{verbatim}
let rec is_less a b =
    match (a,b) with
        (a1::_,b1::_) when a1 < b1 -> true
      | (a1::ax,b1::bx) when a1 = b1 -> is_less ax bx
      | ([],_::_) -> true
      | _ -> false
\end{verbatim}

Но в этом коде нет необходимости, поскольку операции сравнения определены
и для списков, причём списки сравниваются лексикографически.
Например, сравнения \verb![1;2;3] < [2;1;3]!, \verb![1] < [2;1]! и \verb![] < [9;7;10]! вернут \verb!true!.

\subsubsection{Упорядочение перестановок}

Определим порядок на перестановках как лексикографический:

\begin{definition}
Перестановку $\pi$ назовём меньшей перестановки $\sigma$ (и обозначим это
как $\pi\prec\sigma$, если найдётся такой индкес $p$, что:
\begin{enumerate}
\item $\pi(0) = \sigma(0),\quad\pi(1) < \sigma(1),\quad\dots\quad,\pi(p-1)=\sigma(p-1)$
\item $\pi(p) < \sigma(p)$
\end{enumerate}
\end{definition}

\noindent В частности, $p=0$ также можно рассматривать: $[0;1]\prec[1;0]$.

\subsection{Несколько фактов о лексикографическом порядке на перестановках}

\begin{lemma}\label{between}
Рассмотрим перестановки равной мощности $\pi$ и $\sigma$ (пусть для определённости $\pi\preceq\sigma$), 
такие, что некоторый префикс этих перестановок совпадает:
$$\pi(0) = \sigma(0),\quad \pi(1) =\sigma(1),\quad\dots\quad, \pi(p-1) = \sigma(p-1)$$

Тогда для того, чтобы перестановка $\tau$ находилась в лексикографическом порядке
между $\pi$ и $\sigma$ ($\pi\preceq\tau\preceq\sigma$), необходимо, чтобы и она имела
тот же префикс:
$$\tau(0) = \pi(0),\quad \tau(1) = \pi(1) ,\quad\dots,\quad \tau(p-1) = \pi(p-1)$$
\end{lemma}

\begin{proof}
Докажем от противного. Пусть это не так --- тогда найдём минимальный
отличающийся элемент перестановки --- то есть, такой $k$, что $k<p$ и выполнены 
следующие равенства и неравенства:
\begin{enumerate}
\item $\tau(0) = \pi(0) = \sigma(0),\quad \tau(1) = \pi(1) =\sigma(1),\quad\dots,\quad \tau(k-1) = \pi(k-1) = \sigma(k-1)$
\item $\tau(k)\ne\pi(k)=\sigma(k)$
\end{enumerate}
Воспользовавшись определением сравнения перестановок, легко убедиться, 
что в этом случае перестановка $\tau$ либо меньше как
$\pi$, так и $\sigma$ (если $\tau(k) < \pi(k)$), либо больше 
(если $\tau(k) > \pi(k)$). В любом случае мы получаем противоречие с условием.
\end{proof}


\begin{lemma}\label{order}
Рассмотрим множество перестановок длины $n$ с префиксом $[a_0; a_1; \dots; a_{p-1}]$:
$$S^a_n = \{\pi\in S_n | \pi(0)=a_0 \with \pi(1)=a_1 \with\dots\with \pi(p-1)=a_{p-1}\}$$
где $a_i$ --- какие-то заранее заданные значения.

Обозначим минимальную и максимальную перестановки в множестве за $\rho$ и $\sigma$:
$$\rho = \min S^a_n, \quad \sigma = \max S^a_n$$
Тогда справедливо следующее:
$$\rho(p) < \rho(p+1) < \dots < \rho(n-1)$$
$$\sigma(p) > \sigma(p+1) > \dots > \sigma(n-1)$$
\end{lemma}

\noindent Иными словами, минимальная перестановка с данным префиксом --- такая,
в которой все оставшиеся элементы перестановки возрастают, 
а максимальная --- в которой оставшиеся элементы убывают.

Например, $\min S^{[2;4;1]}_7 = [2;4;1;0;3;5;6]$ и $\max S^{[2;4;1]}_7 = [2;4;1;6;5;3;0]$.

\begin{proof}
В самом деле, пусть дана перестановка $\tau \in S^a_n$, покажем, что $\rho\preceq\tau$.
Пусть это не так: то есть $\tau\prec\rho$, то есть найдётся минимальный такой $t$, что 
$$\tau(0) = \rho(0),\quad \tau(1) = \rho(1) ,\quad\dots,\quad \tau(t-1) = \rho(t-1), \quad\mbox{но } \tau{t}<\rho{t}$$

Заметим, что по определению $\tau \in S^a_n$ и $\rho\in S^a_n$, то есть перестановки $\tau$ и $\rho$ совпадают в 
элементах с 0 по $p-1$. Значит, $t\ge p$. 

Может ли $t=p$? Поскольку перестановки имеют одинаковые префиксы, оставшиеся (непрефиксные) множества элементов
совпадают: $$\{\rho(p),\rho(p+1),\dots,\rho(n-1)\} = \overline{0,n-1}\setminus\{a_0,a_1,\dots,a_{p-1}\} =
\{\tau(p),\tau(p+1),\dots,\tau(n-1)\}$$
Однако, $\rho(p) = \min\{\rho(p),\rho(p+1),\dots,\rho(n-1)\}=\min\{\tau(p),\tau(p+1),\dots,\tau(n-1)\}$, то есть $\rho(p)\le\tau(p)$.
По предположению же $\tau\prec\rho$, отсюда неизбежно $\rho(p)=\tau(p)$.

Осталось заметить, что в таком случае мы можем расширить префикс, добавив к нему общий 
для обеих перестановок элемент номер $p$:
$$\tau \in S^{[a_0;a_1;\dots;a_{p-1};\tau(p)]}_n = S^{[a_0;a_1;\dots;a_{p-1};\rho(p)]}_n \ni \rho$$
и, повторив рассуждение выше $n-p$ раз, мы покажем, что либо перестановки $\rho$ и $\tau$ совпадают,
либо в первом различии $\rho(t) < \tau(t)$ (что влечёт $\rho\prec\tau$).

Аналогично можно показать максимальность $\sigma$.
\end{proof}

\begin{lemma}\label{prefixorder}
Рассмотрим множество перестановок $S_n$ и два префикса перестановок одинаковой
длины: $a$ и $b$ ($|a| = |b| \le n$), причём $a \prec b$. Тогда все перестановки 
из $S^a_n$ меньше перестановок из $S^b_n$: $$\forall \pi\in S^a_n\ \forall \tau\in S^b_n\ (\pi \prec \tau)$$
\end{lemma}

\begin{proof}
Очевидно из того, что сравнение перестановок идёт слева направо: сперва мы сравниваем
префиксы, и приступаем к сравнению <<послепрефиксной>> части перестановок только
если префиксы равны. А поскольку префикс $a$ меньше префикса $b$ ($a \prec b$), то и 
перестановки из $S^a_n$ меньше перестановок из $S^b_n$.
\end{proof}

\subsection{Построение следующей перестановки по перестановке из $S_n$}

Анализ вышеуказанных лемм позволяет нам предложить алгоритм построения следующей
перестановки. Пусть дана перестановка $\pi = [a_0; a_1; \dots; a_{n-1}]$. Тогда:
\begin{enumerate}
\item Рассмотрим такое минимальное число $p$, что 
$$a_p > a_{p+1} > a_{p+2} > \dots > a_{n-1}$$
То есть, иными словами, убывающая последовательность имеет максимально возможную
длину. 
\item Среди чисел $a_p, a_{p+1} \dots a_{n-1}$ найдём минимальное, б\'{o}льшее $a_{p-1}$ --- 
пусть это $a_t$.
\item Тогда соберём искомую перестановку из трёх частей:
  $$[a_0; a_1; \dots; a_{p-2}] \ @ \ [a_t] \ @ \ \texttt{sort} (\{a_{p-1},\dots,a_{n-1}\} \setminus \{a_t\})$$
То есть возьмём первые $p-1$ элементов исходной перестановки, дальше добавим $a_t$,
и все оставшиеся элементы перестановки, отсортированные по возрастанию.
\end{enumerate}

\begin{theorem} Предложенный выше алгоритм действительно строит следующую перестановку
в $S_n$, при условии, что $p > 0$.
\end{theorem}
\begin{proof}
Пусть мы по перестановке $\pi$ получили перестановку $\rho$, пользуясь данным алгоритмом.
По лемме \ref{order} $\pi = \max S^{[a_0;a_1;\dots;a_{p-2};a_{p-1}]}_n$
и $\rho = \min S^{[a_0;a_1;\dots;a_{p-2};a_t]}_n$.

По построению, $[a_0;a_1;\dots;a_{p-2};a_{p-1}] \prec [a_0;a_1;\dots;a_{p-2};a_t]$, отсюда
по лемме \ref{prefixorder} все элементы из $S^{[a_0;a_1;\dots;a_{p-1}]}_n$ меньше 
элементов из $S^{[a_0;a_1;\dots;a_{p-2};a_t]}_n$, значит, $\pi \prec \rho$.

По лемме \ref{between} между перестановками $\pi$ и $\rho$ нет перестановок с префиксом,
отличным от $[a_0;a_1;\dots;a_{p-2}]$. Каким же может быть $p-1$ значение в префиксе?
Оно не может быть меньше $a_{p-1}$ (иначе перестановка меньше $\pi$) и не может быть
больше $a_t$ (иначе перестановка больше $\rho$). А между $a_{p-1}$ и $a_t$ значений
нет. При этом $\pi$ --- максимальная перестановка с префиксом $[a_0;a_1;\dots;a_{p-2};a_{p-1}]$,
а $\rho$ --- минимальная с префиксом $[a_0;a_1;\dots;a_{p-2};a_t]$.
Значит, $\pi$ и $\rho$ --- соседние.
\end{proof}

%\subsection{Сравнение перестановок разной длины}
%
%Чтобы определить сравнение перестановок разной длины, определим вложение более коротких
%перестановок в более длинные.
%
%\begin{definition} Пусть $\pi\in S_n$ и $\rho\in S_k$. Пусть $n>k$. Тогда 
%перестановки $\pi$ и $\rho$ эквивалентны, если $$\pi(t) = \left\{\begin{array}{ll}\rho(t),&t<k$\\t,&t\ge k\right.$$
%\end{definition}
%\noindent Иными словами, перестановки эквивалентны, если они одинаково переставляют общие элементы, и не
%трогают элементы, которые не являются общими.
%
%Например, перестановки $[1;0]$ и $[0;1;2;3;4;6;5]$
%
%\begin{definition}
%Пусть даны две перестановки, $\pi\in S_n$ и $\rho\in S_k$.
%Будем говорить, что $\pi\prec\rho$, если найдутся такие эквивалентные перестановки $\pi'$ и $\rho'$ из
%$S_{\max(n,k)}$, что $\pi'\prec\rho'$.
%\end{definition}
%
%Несложно показать, что данное определение не противоречит ранее данному определению для 
%
%\subsection{Сопоставление факториального числа и перестановки}
%
%Рассмотрим перестановки мощности $n$. Заметим, что в силу лексикографического порядка 
%перестановок, все 
%
%
%Однако, если мощности различаются, всё становится сложнее ---
%
%\begin{definition}Пусть $|\sigma| < |\pi|$, и пусть $k = |\pi|-|\sigma|$. То есть
%перестановка $\pi$ мощнее перестановки $\sigma$ на $k$ элементов. 
%Назовём перестановку $\sigma$ частным случаем перестановки $\pi$, если выполнены сразу следующие два условия:
%\begin{enumerate}
%\item $\pi_0 = 0, \pi_1 = 1, \dots, \pi_{k-1} = k-1$
%\item $\pi_k = \sigma_0, \pi_{k+1} = \sigma_1, \dots, \pi_{k+|\sigma|-1} = \sigma{|\sigma|-1}$
%\end{enumerate}
%\end{definition}
%
%Столь сложное определение вызвано тем, что сравниваем перестановки мы справа налево
%(наименее значимый разряд --- крайний справа), однако, нумеруем мы разряды справа налево
%(элемент $0$ --- крайний слева). 
%
%Поэтому нам нужно выровнять перестановки для правильного сравнения.
%
%\subsection{Факториальные числа}
%
%Если же все буквы одинаковы, то меньше то слово, которое короче (слово \emph{БАЛ} идёт раньше 
%слова \emph{БАЛКА}). 

\end{document}