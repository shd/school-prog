Перечисление объектов

Решение очень многих задач начинается с перечисления каких-либо объектов.
Вспомните, например, что мы делаем, если решаем уравнение методом подбора 
решений: мы сперва перечисляем все возможные значения, потом их проверяем 
одно за одним, и отбрасываем лишние. Построение таблиц (да хотя бы обычной
таблицы умножения), списков, подведение каких-нибудь итогов --- всё это 
начинается с перечисления каких-то (начальных) значений, которые потом уже
дополнительно преобразуются. 

В этой главе будем перечислять числа в различных системах счисления и всякие 
родственные объекты (перестановки, сочетания и т.п.), поскольку,
с одной стороны, эти объекты перечислать сравнительно просто, а, с другой
стороны, множество приёмов и подходов видно уже здесь.

Построение списков 

Построение списков целых чисел

Это задача уже давно известная, мы можем только повторить классическое решение:
можем, например, написать рекурсивную функцию \verb!gen_range: int -> int -> int list!,
которая берёт два аргумента (начальное и конечное значения), и возвращает
список:

\begin{verbatim}
let rec gen_range start end =
    if start > end then []
    else start :: gen_range (start+1) end;;
\end{verbatim}

Однако, всё меняется, как только мы меняем представление двоичных чисел.