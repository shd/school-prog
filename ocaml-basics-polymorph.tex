\subsection{Полиморфные типы}

В предыдущем разделе мы старательно обходили стороной примеры вроде
такого:

\begin{verbatim}
let max a b = if a > b then a else b;;
\end{verbatim}

Если мы разберемся в типах в этой функции, мы максимум сможем
установить, что два параметра данной функции имеют одинаковый тип и 
результат имеет тот же самый тип. Тем не менее, язык Окамль позволяет
такие требования на типы формализовывать с помощью \emph{полиморфных}
типов.

Сигнатура функции \s{max} будет такой: \s{'a -> 'a -> 'a}. 
Тип \s{'a} называется <<типовой переменной>> и означает, что в данном
месте может быть указан любой тип. Скажем, если функцию \s{max} применить
к двум аргументам типа \s{string} (и, соответственно, \emph{связать}
типовую переменную \s{'a} с типом \s{string}), то результат тоже будет
иметь тип \s{string}.

